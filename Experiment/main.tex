\documentclass{resume}
\usepackage[left=0.3 in,top=0.5in,right=0.3in,bottom=0.2in]{geometry}
\newcommand{\tab}[1]{\hspace{.2667\textwidth}\rlap{#1}}
\newcommand{\itab}[1]{\hspace{0em}\rlap{#1}}
\usepackage{hyperref}
\usepackage{enumitem}
\name{Kausik}
\address{\href[pdfnewwindow=true]{mailto:ucangetkausik@gmail.com}{ucangetkausik@gmail.com} \addressSep \href{https://github.com/Kausik-A}{github.com/Kausik-A} \addressSep \href{https://www.linkedin.com/in/kausik-a/}{linkedin.com/in/kausik-a} \addressSep \textnormal{Seattle, WA}}
\begin{document}
   \vspace{-5.5pt}
\begin{rSection}{Work Experience}

  \small
    {\bf Amazon Inc : Software Development Engineer } \\
  {GoLang, Java, terraform, AWS lambda, kubernetes}\hfill {\sl Feb 2022 - Present}
    \begin{itemize}[itemsep=-7pt,topsep=-3pt]
    \item Working on Amazon distro for opentelemetry (\href{https://aws.amazon.com/otel/}{ADOT}) under AWS observability org and opentelemetry.
    \item Helped deploy the observability agent as an service internally and scaled it up to handle over 500M requests/min with 99.8\% SLA . Also helped various external customers on deployment and scaling issues.
    \item Contributed to open source observability tools under opentelemetry by building CI/CD pipelines using github actions, terraform and AWS cloud-formation. This also included building validation tools in java.
    \item Worked on building language framework SDKs for the observability agent. Also am a maintainer for the opentelemetry python SDK libraries.
    \end{itemize}
  	\small
    {\bf Research Assistant : Mobile and Robotic Systems Lab} \\
  {Embedded Systems, Azure Cloud, IOT, Data Pipelining} \hfill {\sl May 2021 - Dec 2021}
    \begin{itemize}[itemsep=-7pt,topsep=-3pt]
    \item Collaborated with researchers from Departments of CS, GeoScience, and Physics on \textbf{Radon Detection testbed Research} (\href{https://news.gsu.edu/2021/03/30/georgia-state-research-team-receives-1-2-million-grant-to-develop-novel-radon-testing-system-in-metro-atlanta/}{NSF Grant})
    \item Investigated Radon propagation by engineering a Sensor nest ( Raspberry Pi,Arduino )that measures various parameters at different topographies.
    \item Designed a data pipeline to store all live result data asynchronously on \textbf{Azure cloud using Azure -  IOT edge and stream analytics.}
    \item Upscaled 200 nest devices with \textbf{100\% data redundancy} through a 3-layer data storage architecture, obtained \textbf{99.8\% accuracy.}
    \end{itemize}
  	\small
    {\bf Facebook EIR : Teaching Assistant  } \\
    {CSC 4520 (\href{https://web.archive.org/web/20210423114626/https://csc4520.org/}{csc4520.org}), Algorithms,Data Structures, Java} \hfill {\sl Jan 2021 - April 2021}
    \begin{itemize}[itemsep=-7pt,topsep=-3pt]
    \item Facilitated a class of 170 students, teaching intricate concepts of \textbf{algorithms} to help students analyze and solve problems.
    \item Assisting students during Office Hours, Code reviewing and assessing their assignments through Github pull requests, and actively clarifying queries on Piazza.
    \item Streamlined grading methodology using Github auto graders, Java Unit Testing, shell scripts, and Python Pandas \textbf{reduced manual interference by 70\%.}
    \end{itemize}
	\small
    {\bf Research Intern : Defence Research and Development Organisation, India (DRDO)} \\
    {Python, OpenCV, Deep Neural Networks, Secure AI} \hfill {\sl May 2019 - July 2019}
    \begin{itemize}[itemsep=-7pt,topsep=-3pt]
    \item Collaborated on the project - \textbf{Facial Recognition System} with the Advanced Systems Laboratory (ASL) team, devised a model using deep neural network architecture in OpenCV to detect potential intruders with an \textbf{accuracy of 91\% }
    \item Programmed an optimal algorithm for building robust AI through Adversarial training and encrypting employee data.
    \end{itemize}


  \end{rSection}
  \vspace{3pt}
 \begin{rSection}{Technical Skills}
  	\small
       \begin{tabular}{ @{} >{\bfseries}l @{\hspace{6ex}} l}
      Languages &  Golang, Java, Python, JavaScript, SQL, Kotlin \\
      \vspace{1pt}
      Frameworks \& Libraries &   Spring Boot, React, Spark, Pandas, NumPy \\
      \vspace{1pt}
      Cloud \& Infrastructure & AWS (Lambda, CloudFormation, ECS/EKS), Azure (IoT Edge, Stream Analytics), \\
      & Terraform, Docker, Kubernetes \\
      \vspace{1pt}
      Tools \& DevOps & Git, GitHub Actions, OpenTelemetry, Prometheus, Grafana, CI/CD Pipelines \\
      \vspace{1pt}
      Databases & DynamoDB, MySQL, MongoDB
    \end{tabular}
  \end{rSection}
  \vspace{3pt}
\begin{rSection}{Education}
  	\small
    {\bf Master of Science in Computer Science}  \hfill {\sl Aug 2020 - Dec 2021}\\
    {Georgia State University, Atlanta}\\
    {\bf Bachelor of Technology - Computer Science and Engineering}  \hfill {\sl Aug 2016 - May 2020}\\
    {GITAM University, India}

  \end{rSection}
  \vspace{3pt}
  \begin{rSection}{Projects}
  \small
    {\bf Apple News Forum} {\hfill \sl \href{https://www.applegossips.com}{applegossips.com}}
    \begin{itemize}[itemsep=-7pt,topsep=-3pt]
     \item Co-founded an Apple news outlet. Developed using Node.js, MongoDB and CRUD REST APIs
    \end{itemize}
     {\bf Querying Student data using Virual Assistant}
    \begin{itemize}[itemsep=-7pt,topsep=-3pt]
    \item Remodeled a Virtual Assistant that queries student data from the university's website to give the user vocal information. Built by web scraping using Beautiful-soup, XML Parsing, REST API which is deployed on Google Cloud platform.
    \end{itemize}
  \end{rSection}
\end{document}
